\documentclass[letter,11pt]{article}
\usepackage{cv}
\begin{document}
    \begin{justify}
        I am an embedded software developer with an interest in aerospace, electrical and civil engineering. My past experiences include CubeSat onboard electronics design, flight control software development and mechanical simulations. I also had some experience in structural strength assessment for rockets during my internship at V.P.Glushko JSC Energomash.
        
        \setlength{\parskip}{1em}
        \setlength{\parindent}{0em}
        Most of my past and current assignments involved working with CAD software --- for mechanics or electronics. While using them for electronics or structural design exclusively, usually workflows are straightforward and hassle-free. But when there is a need to inter-operate EDA (electronic design automation, or ECAD) software with structural design software, for example, to simulate stresses on a PCB during rocket ascent, things are getting complicated. Most EDAs accomplish inter-operability with structural CADs by exporting PCB designs to file formats such as STEP or VRML. These formats do not preserve the structural properties of PCBs and do not include material properties like mass and density. It becomes problematic to simulate structural deformations and calculate inertial parameters, which is very important in the aerospace industry. 

        I faced all of these problems when I designed CubeSat payload PCB in KiCad and then tried to simulate it in SolidWorks. I am applying to the National Taiwan University Department of Civil Engineering to improve the situation with electronics-structural CAD integration, and to develop new tools for PCB structural strength assessment.
   
        In NTU, I plan to take courses in the field of civil engineering (Criteria F-1, Civil Engineering Group, and all other courses required to graduate).
    \end{justify}
\end{document}




