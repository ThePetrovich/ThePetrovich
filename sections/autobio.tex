\documentclass[letter,11pt]{article}
\usepackage{cv}
\begin{document}
    \begin{justify}
    When I was a child and studied in the first grade of elementary school, I saw a very bright star in the sky. It was moving quickly and disappeared in a matter of seconds, almost like a meteor. Frightened, I asked my teacher about this, and he told me that it came from one of the Iridium satellites. I was still too young to understand what an ''Iridium satellite'' is, but it calmed me down, and I quickly forgot about it.
    \setlength{\parskip}{1em}
    \setlength{\parindent}{0em}

    Then I saw it again 2 or 3 years later. I googled the hell out of this topic, didn't understand anything... until I stumbled upon a game called Kerbal Space Program. Yeah. I bet every aerospace engineer born after the year 2000 has the same backstory: minding their business $\rightarrow$ discovering KSP by accident $\rightarrow$ bargaining $\rightarrow$ having fun $\rightarrow$ job at JPL.

    I am currently in the middle of the ''having fun'' phase, and in the following few paragraphs, I will briefly describe how I got there and how I will move to the ''job at JPL''.
    First, ''bargaining''. When playing KSP for the first time, you can have two outcomes: you either crash your rocket, question your qualification, close the game and never open it again or you /still/ crash your rocket but become obsessed with it. In my case, it was obviously the latter, so I went to our local rocketry club to explode rockets IRL. Still remember my first launch: 4th grade, -40°C outside, getting plasticine Jebediah stuck on a tree. I still have this rocket lying in my garage.

    ''Crashing paper rockets is fun, but crashing paper rockets with EXPENSIVE EQUIPMENT should be even better!'' - I thought when enrolling in the CanSat team in early high school. It was also when I started learning programming and electrical engineering in depth. If you don't know, CanSat is a small model satellite launched on a rocket up to 1 km, then recovered on a parachute. I was responsible for the flight software and data handling. Most people I know hated it because you had to code in pure C and do some shady stuff with low-level configuration, but for me, it was the best thing ever! Seeing a bunch of random hardware running code written by myself entirely still blows my mind. I'm not even regretting countless hours of debugging.

    Our team saw moderate success in competitions, and soon I was assigned the team leader role. I was now in 10th grade, and before graduating, we decided to pull off something crazy: launching a CubeSat, designed and built by us. Spoiler: we couldn't (2020 happened), but some of our work is indeed flying in space. After successful participation in the <<Big Challenges>> all-Russian project competition, our team was invited to the Sirius Center, where we calibrated an on-board camera for the 3U CubeSat <<CubeSXSiriusHSE>>. It was put in orbit on 26th March 2021.

    2020 hit us hard, and I could not leave Yakutsk, so I decided to take a gap year (or rather two gap years) and finish some of my projects. During that time, Sakha Junior Science Academy hired me to design a payload for their CubeSat, which brings us to the present. I am currently holding the position of Electrical Engineering Lead, and I am pleased to announce that our satellite is flying in December 2022. We already got all the papers required, the satellite is now in manufacturing, we got funding, and I got an official qualification in spacecraft design. In other words, things are going well.
    \end{justify}
\end{document}




